\section{Waiting for the Dots}
\label{dots_explained}
For anybody who played \doom{} on a PC in 1994, the most frustrating part was to wait for the game to load. One step in particular, the mysterious \cw{R\_Init},  seemed to take forever\footnote{In fact, it took anywhere from 15 to 30 seconds depending on the HDD speed.}. An improvised progress bar made of dots helped the player to be patient. Millions of hours were spent watching these tidy dots progress to the right. And probably a few more trying to guess what \cw{R\_Init} actually did in the background.\\
\par
\fakedosoutput{dots.txt}\\
\par
With access to the source code it is possible to modify the engine to output a "label" matching the current phase performed instead of dots. It turns out there are eleven "phases".\\
\par
\fakedosoutput{dots_explained.txt}\\
\par

Phase \cw{OOOOO} correspond to \cw{R\_InitTextures}. This is where the texture definitions are read and kept in RAM. Something that was not mentioned in the 3D renderer section of the book (for the sake of simplicity) is that texture are made of patches. This is done in order to save space in the case of repeating pattern picked by the designer.\\
\par
The first step of this phase is to open a lump named \cw{PNAMES} which contains all the patches names. From the order they appear is generated a mapping [patch name, ID].\\
\par
The second step (which represent the bulk of the processing time) is to open lumps \cw{TEXTURE1} and \cw{TEXTURE2}. These contain all texture entries. Each entry features a name and a list of patches IDs along with patch coordinate and offsets. Upon drawing a wall, \doom{} lookup the texture name and retrieves all patch texels by retrieving lumps based on patch names.\\
\par
Phase \cw{1} is just a marker showing when \cw{R\_InitTextures} returns.\\
\par

Phase \cw{2} corresponds to \cw{R\_InitFlats}. It looks for lumps \cw{F\_START} and \cw{F\_END} which are the marker surrounding flat textures. Only the number of flats is retrieved so a proper \cw{malloc} for the flat array can be performed.\\
\par

Phase \cw{333333333333} is similar to phase three but this time it involves sprites lumps. Function \cw{R\_InitSpriteLumps} looks up lumps \cw{S\_START} and \cw{S\_END} to find the width and hoffset of all sprites in the wad  and save that data in a sprite array. In the process it prints a dot every 64 sprites. This was the slowed phase in \cw{R\_Init} since it had to read a lot of data.\\
\par

Phase five (\cw{4}) is a simple marker to show the end of \cw{R\_InitSpriteLumps}.\\
\par

Phase six (\cw{5}) is also a marker to show end of \cw{R\_InitData} which encompass phases \cw{1}, \cw{2}, \cw{333333333333}, and \cw{4}.\\
\par

Phase \cw{6} matches function \cw{R\_InitPointToAngle} when used to be built the tangent lookup table. This is now an empty function since is it pre-calculated and stored in \cw{tables.c}\\
\par

Phase \cw{7} matches function \cw{R\_InitTables}. This is where used to be built lookup tables \cw{finetangent} and \cw{finesine}. Like the tangent lookup table, there are no pre-calculated in \cw{tables.c} and baked in the executable.\\
\par

Phase \cw{8} matches function \cw{R\_InitPlanes} and does nothing. What a waste of a dot.\\
\par

Phase \cw{9} matches function \cw{R\_InitLightTables} and initialize the zlight table used to implement the lightmaps.\\
\par

Phase \cw{A} matches function \cw{R\_InitSkyMap} which initialize the static \cw{skyflatnum}.\\
\par
As you will have guessed, the "dot thermometer" was not very accurate. It started fast, slowed down for \cw{333333333333} which access a lot of data, and then sped up again to the end while calling mostly empty functions.