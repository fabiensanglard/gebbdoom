This is the second book in the \textit{Game Engine Black Book} series. It picks up right where the first one ended with the release of Wolfenstein 3D in May 1992. This volume describes the engineering history of a tile which was also developed by id Software. It took them only 11 months and produced what is universally considered one of the best game of all times: \doom.\\
\par
It may seem odd to write a book about a game twenty-five years after its release. Who would read about a seemingly outdated technology dedicated to extinct hardware? It turns out there are many reasons. For one, \doom has had such a profound and sustained impact that it has become part of computing history. An unquestionable milestone, entertaining millions and triggering countless careers. Because the source code was officially released, many programmers have learned to code by looking under the hood. Because it was easy to modify, countless\footnote{Often now professional game designers.} started to design levels or generating new assets. Because it was easy to port to a new system, to this day it is often the first "tour de force" for a programmers\footnote{From MacBook TouchBar, fridges to ATM, \doom has been ported to absolutely everything with a screen and a CPU.} willing to demonstrate his skills.\\
\par
Beyond the sentimental value, this is the story of a team of people who gathered around a common vision. There was no clear path between where they were and where they wanted to be. Only the certitude that nobody else had done it before. They worked days and nights, sometimes sleeping on floors\footnote{Or couch in the case of Dave Taylor} to make their dream come true. This is the never ending story of inventors, engineers and builders. Alike building pyramids or bringing men on the moon, the story of the making of \doom describes well how acheiving something big assimilates to do a thousands small things right.\\
\par
In the case of \doom, a fistful of people carefully crafted new tools to create breathtaking engine, design, and assets resulting in the perfect storm. They released something of such great quality, it is still praised and purchased 25 years later.\\
\par
 To tell this story, the author faced two seemingly orthogonal constraints. On one side the desire to have this book stand alone, without need for supplemental information or cross-reference to previous books. On the other side, avoid boring faithful readers with content described in previous entries.\\
\par
Although content re-usage was kept to a minimum, ultimately, new readers were favored. It would have been a frustrating experience to constantly direct somewhere else for missing knowledge. The book would have felt incomplete.\\
\par
I hope readers with previous knowledge from \textit{Game Engine Black Book: Wolfenstein 3D} will forgive being faced with VGA architecture description, floating point internals, and i386 segmented memory again in the hardware chapter. It is not a stratagem to increase the page count. There is much truth in "Less is more" and 300 pages would have been preferable to the 400ish of this volume but the complexity of the engine code and love for details prevented it.\\
\par
-- Fabien Sanglard (fabiensanglard.net@gmail.com)