This is the second book in the \textit{Game Engine Black Book} series. It picks up right where the first one ended with the release of Wolfenstein 3D in May 1992. It carries on all the way down to December 1993 with the second breakthrough of the 90s in the world of PC gaming, \doom.\\
\par
 Like its predecessor, this volume attempts to describe in great details both the hardare and the software of the era. It opens a window back in time peeking over the engineering used to solve the various problems id Software had to face during the eleven months its took them to ship their new title.\\% which resulted in what is universally considered one of the best game of all times.\\
\par
It may seem odd to write a book about a game twenty-five years after its release. After all, who would be interested in a seemingly outdated technology dedicated to extinct hardware running obsolete operating systems? I believe there are many aspects in the making of \doom which carry important historic, nostalgic, engineering, industrial and philosophic values.\\ 

\par
First, \doom has had such a profound and sustained impact that it has become part of modern history. It was an unquestionable milestone which entertained millions and catalyzed vocations. Because the source was available, programmers have learned to hack with it. Because it was easy to modify, countless aspiring game makers first started by making levels or drawing assets. To this day, because it is such an icon, it is the goto title for a hacker willing to demonstrate their skills. From MacBook Touchbar, ATMs, CT scanner, watches and even fridges\footnote{There is even a website "itrunsdoom.tumblr.com" tracking what \doom has been ported to.} pretty much any piece of electronic has run \doom.\\
\par

Second, it was a financial and critical success which reshaped the PC gaming industry. During 1994 it received many awards, including \textit{Game of the Year} by both \textit{PC Gamer} and \textit{Computer Gaming World}, \textit{Award for Technical Excellence} from \textit{PC Magazine}, and \textit{Best Action Adventure Game award} from the \textit{Academy of Interactive Arts \& Sciences}. With more than two millions unit sold and an estimated 20 millions shareware installation, at its highest it generated close to \$100,000 per day. Before the term was overtaken by "First Person Shooter" people talked about "doom-like" games.\\
 \par
 \rawpngdrawing{Doom_clone_vs_first_person_shooter}
 \par
There is also the non-negligible sentimental value. \doom is one of these title which made an ever lasting impression upon first contact. Many of us were just in our teen years when this game came out and most are still able to remember in which circumstances they first saw it running. I can vividly remember going over to the neighbor's house (lucky owned of a 486) and thinking it was the coolest thing he had ever seen (after Chris Magic Waddle of course).\\
\par


Beyond the nostalgia, and this is the most important reason this book is relevant, the making of \doom is the ever repeating story of inventors, engineers and builders gathered around a common vision. There was no clear path between where the id Software team was and where they wanted to be. Only the certitude that nobody else had gone there before. They worked days and nights, slept on the floor\footnote{Or couch in the case of Dave Taylor.} to make their dream come true. Alike building pyramids or bringing men on the moon, \doom happening summarizes perfectly how achieving one big goal assimilates to do a thousands small things right. This is the story of a bunch of dreamers who had the skills, dedication, and good fortune to take then all the way resulting in a perfect storm of game engine, artwork and design.\\
\par



You may agree or not with the values of these "old" things. Some people prefer to sail with the wind, rarely looking back. But even to them this book could turn out to be a useful engineering map someday.\\
\par
 This story had to respond to two seemingly orthogonal constraints. On one side it had to stand alone without need for supplemental information or cross-reference to previous books. On the other side, it had to avoid boring faithful readers with content already visited in the series. The middle ground was to mention non-essential aspects but not dig too much. As a result, the architecture of the VGA, Real-Mode, PC Speaker sound synthesis, PIC and PIT, and a few other topics are mentioned but not extensively described. This trade-off allowed to reach the target which was a booklet less than 400 pages which can handle with one hand while sipping tea.\\
\par
I hope you will enjoy it.\\
\par
-- Fabien Sanglard (fabiensanglard.net@gmail.com)