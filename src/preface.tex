This is the second book in the \textit{Game Engine Black Book} series. It picks up right where the first one ended with the release of Wolfenstein 3D in May 1992. There was no hesitation about what game should be documented next. It obviously had to be \doom. This volume describes the engineering history from early 1993 to Christmas of the same year. It details the problems and solution id Software team encountered during the eleven months it took them to produce what is universally considered one of the best game of all times.\\
\par
It may seem odd to write a book about a game twenty-five years after its release. Who would read about a seemingly outdated technology dedicated to extinct hardware? It turns out there are many reasons, each would have warranted a book on their own.\\ 
\par
First, \doom has had such a profound and sustained impact that it has simply become part of modern history. An unquestionable milestone, entertaining millions, shaping the industry, and steered entire careers. Because the source was available, programmers have learned to code by looking under the hood. Because it was easy to modify, countless contemporary designers first started by designing levels or drawing assets for \doom. Even to this day, because it is so easy to port, it has become the goto title for a hacker willing to demonstrate his skills to get it to run on her favorite piece of electronic. From a MacBook Touch Bar, ATMs, CT scanner, watches and even fridges\footnote{\doom has been ported to absolutely everything with a screen and a CPU, there is even a website "itrunsdoom.tumblr.com" tracking it.}.\\
\par
Beyond the sentimental value, \doom story is the ever repeating story of inventors, engineers and builders gathered around a common vision. There was no clear path between where they were and where they wanted to be. Only the certitude that nobody else had done it before. They worked days and nights, slept on the floor\footnote{Or couch in the case of Dave Taylor.} to make their dream come true. Alike building pyramids or bringing men on the moon, \doom happening summarizes perfectly how achieving something big assimilates to do a thousands small things right. A fistful of people carefully crafted new tools to create breathtaking engine, design, and assets resulting in the perfect storm. They released something of such great quality, it is still praised and purchased twenty five years later.\\
\par
 To tell this story, the author faced two seemingly orthogonal constraints. On one side the desire to have this book stand alone, without need for supplemental information or cross-reference to previous books. On the other side, avoid boring faithful readers with content described in the previous entry.\\
\par
Although content re-usage was kept to a minimum, ultimately, new readers were favored. It would have been a frustrating experience to constantly direct somewhere else for missing knowledge. The book would have felt incomplete. I hope readers with previous knowledge from \textit{Game Engine Black Book: Wolfenstein 3D} will forgive being bothered with VGA architecture, floating point internals, and i386 segmented memory. It is not a stratagem to increase the page count. There is much truth in "Less is more" and 300 pages would have been preferable to the 500 of this volume but the complexity of the engine code and love for details prevented it.\\
\par
-- Fabien Sanglard (fabiensanglard.net@gmail.com)