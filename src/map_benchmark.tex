 As of 2018, it has been twenty five years since the last machine came out of \NeXT{}'s Redwood City factory in 1993. It has become a rare occurrence to find one of these pieces of black hardware in working condition.\\
 \par
 Since this book strives to be historically accurate, it was paramount to find an actual NeXTstation TurboColor -- first and foremost to document the development condition of the time, but also to witness the full game pipeline in motion. Even though passionate and dedicated programmers have produced a gorgeous emulator called "Previous", the performance numbers would not have been accurate.\\
 \par
  I lucked out on eBay and found exactly the configuration I needed. The machine was in working condition but the SCSI hard-drive was making clinking noises, a sign that it was about to die. Additionally, the MegaDisplay colors had faded out\footnote{This would have been easily fixable by replacing the capacitors on the monitor control board, but it would not have solved the issue of the weight.} and its 50 lbs (23 kg) made it difficult to move.\\
  \par
  Thanks to Rob Blessin, owner and founder of Black Hole Inc., I was able to replace the HDD with a SD card SCSI2SD providing similar access time. It was difficult to find a screen compatible with \NeXTns{}'s exotic "sync on green" but thanks to the wonderful people at \cw{www.nextcomputers.org} I was pointed to a NEC MultiSync 1980SX which worked flawlessly.\\
  \par
  Words cannot convey how it felt to hear the humming of the machine's fan. To witness \cw{Doom.app}, \cw{DoomED}, and \cw{doombsp} compile flawlessly. To witness this NeXTstation Turbo Color (serial \cw{\#ABC0053943}) come back to life. The machine did not cure cancer as Jobs wished for but it did provide happiness to countless developers.





\section{Developing The Game}
On this double page is recreated the typical developer desktop setup. Notice "Interceptor VGA Console" which gives away \cw{libinterceptor.a}, a private library provided by \NeXTns's engineers to punch a hole in Display Postscript and bypass the "slow" compositor.\\
\par
\cscaledimage{0.9}{doom_on_next.png}{NeXTSTEP development setup (left part of the screen)}

The MegaDisplay resolution of 1120x832 was so high that id Software had to implement a 2x software zoom for the game window. Without it, the \doom{} window looked like a tiny stamp with barely any pixels visible.\\
\par
\vspace{18.5pt}
\cscaledimage{0.92}{doom_on_next2.png}{NeXTSTEP development setup (right part of the screen)}
\pagebreak

\section{Compiling Maps}
Benchmarks for \cw{doombsp} run time for each level in \doom~and \doomii\footnote{Based on \cw{.map} files released by John Romero on 2015-04-22.}.\\
\par
 \begin{minipage}[t]{0.45\textwidth}
 \begin{figure}[H]
\centering  
\begin{tabularx}{\textwidth}{ L{0.3} | R{0.7} }
  \specialrule{1pt}{0pt}{0pt}
  \textbf{Map} & \textbf{\cw{doombsp} runtime (s)} \\
  \specialrule{1pt}{0pt}{0pt}
E1M1 &     8.2 \\ 
E1M2 &       32.0 \\
E1M3 &       26.2\\
E1M4 &       18.4\\  
E1M5 &       19.9\\
E1M6 &       44.0\\
E1M7 &       22.3\\
E1M8 &        6.9\\
E1M9 &       15.4\\
E2M1 &        6.0\\
E2M2 &        55.4\\
E2M3 &        19.6\\
E2M4 &        36.0\\
E2M5 &        46.8\\
E2M6 &        32.5\\
E2M7 &        60.8\\
E2M8 &         2.5\\
E2M9 &         1.5\\
E3M1 &        2.5\\
E3M2 &        9.2\\
E3M3 &       38.1\\
E3M4 &       23.7\\
E3M5 &       34.5\\
E3M6 &       22.5\\
E3M7 &       23.4\\
E3M8 &        1.9\\
E3M9 &        8.9\\
   \specialrule{1pt}{0pt}{0pt}
\end{tabularx}
%\caption{Video system interface}
\end{figure}
\end{minipage}
\hspace{1cm}
\begin{minipage}[t]{0.45\textwidth}
 \begin{figure}[H]
\centering  
\begin{tabularx}{\textwidth}{ L{0.3} | R{0.7} }
  \specialrule{1pt}{0pt}{0pt}
  \textbf{Map} & \textbf{\cw{doombsp} runtime (s)} \\
  \specialrule{1pt}{0pt}{0pt}
MAP01 &       6.1  \\
MAP02 &       6.6 \\
MAP03 &       8.7 \\
MAP04 &       8.5 \\
MAP05 &       17.6\\
MAP06 &       25.0\\
MAP07 &       1.9 \\
MAP08 &       15.2\\
MAP09 &       16.3\\
MAP10 &       34.0\\
MAP11 &        15.7 \\
MAP12 &        15.2\\
MAP13 &        31.5\\
MAP14 &        44.7\\
MAP15 &        66.0\\
MAP16 &        16.2\\
MAP17 &        36.2\\
MAP18 &        17.2\\
MAP19 &        45.8\\
MAP20 &        29.2\\
MAP21 &        5.7 \\
MAP22 &        9.4 \\
MAP23 &        7.5 \\
MAP24 &       30.5 \\
MAP25 &       21.1 \\
MAP26 &       18.8 \\
MAP27 &       26.2 \\
MAP28 &       19.6 \\
MAP29 &       45.8 \\
MAP30 &        1.0 \\
MAP31 &       16.4 \\
MAP32 &        2.7 \\
MAP33 &        6.6 \\
MAP34 &        9.3 \\
MAP35 &        0.3 \\
   \specialrule{1pt}{0pt}{0pt}
\end{tabularx}
%\caption{Video system interface}
\end{figure}

\end{minipage}
\\

\section{Running The Game}
Running \doom{} on a NeXTstation TurboColor produced a surprisingly poor framerate.\\
\par
 \begin{figure}[H]

\centering  
\begin{tabularx}{\textwidth}{ L{0.08} | C{0.42} | R{0.25} | R{0.25} }
  \specialrule{1pt}{0pt}{0pt}
  \textbf{Mode} & \textbf{Resolution} & \textbf{High Details FPS} & \textbf{Low Details FPS} \\
  \specialrule{1pt}{0pt}{0pt}
B & 320x200 & 9 & 13 \\  
A & 320x168&  9 & 14 \\
9 & 288x144& 11 & 15 \\
8 & 256x128& 12 & 16 \\
7 & 224x112& 13 & 17\\
6 & 192x096& 15 & 19 \\
5 & 160x080& 17 & 20 \\
4 & 128x064& 19 & 22 \\
3 & 096x048& 21 & 23 \\
   \specialrule{1pt}{0pt}{0pt}
\end{tabularx}
\caption{\protect\doom{} framerate on a NeXTstation TurboColor}
\end{figure}
\par
It gets even worse when running the game with the 2x zoom used during development. In this mode, the same number of pixels are written to the core's framebuffer but four times more data must transit over the bus.\\
\par
 \begin{figure}[H]
\centering  
\begin{tabularx}{\textwidth}{ L{0.08} | C{0.42} | R{0.25} | R{0.25} }
  \specialrule{1pt}{0pt}{0pt}
  \textbf{Mode} & \textbf{Resolution} & \textbf{High Details FPS} & \textbf{Low Details FPS} \\
  \specialrule{1pt}{0pt}{0pt}
B & 640x400 & 6 & 8 \\  
A & 640x336 & 6 & 8 \\
9 & 576x288 & 7 & 9 \\
8 & 512x256 & 8 & 9 \\
7 & 448x224 & 8& 10\\
6 & 384x192 & 9 & 10 \\
5 & 320x160& 9 & 10 \\
4 & 256x128& 10 & 11 \\
3 & 192x096 & 11  & 11 \\
   \specialrule{1pt}{0pt}{0pt}
\end{tabularx}
\caption{\protect\doom{} 2x zoom framerate on a NeXTstation TurboColor}
\end{figure}
\par
Lowering the resolution or the detail level helped a little bit but not as much as with the DOS version. That's because on \NeXTns{}, the video system is implemented differently.\\
\par
 The implementation disregards update signals from \cw{I\_UpdateNoBlit} and defers all work to \cw{I\_FinishUpdate} where the full content of framebuffer \#0 is blitted to the \cw{NSWindow}. There is no dirty box optimization and no direct access to the hardware like on DOS.\\
\par
\trivia{The "low detail" mode was never properly implemented on NeXTSTEP. The engine writes only half the columns but there is no system to duplicate them like the VGA bank mask did. As a result, only the left portion of the \cw{NSWindow} is updated.}

\section{Framebuffer Non-distortion}
The NeXTstation had a "clean" video system where the color space was linear and the pixels were "square" (the framebuffer had the same aspect ratio as the MegaDisplay monitor). As a result \doom{}'s' framebuffer \#0 suffers no distortion when presented by the window system.\\
\par
The 320x200\footnote{320x200 is the dimension of the active area, not including the title bar.} "Interceptor VGA Console" \cw{NSWindow} is not stretched to 320x240 and therefore appears vertically squashed. It is particularly noticeable when the splash screen on NeXTSTEP (Figure \ref{doom_crushed}) is displayed next to the DOS version (Figure \ref{doom_title_4_3}).\\
\par
\cscaledimage{1}{doom_crushed}{\doom{} on NeXTSTEP. Content appears vertically squashed}
\par
\trivia{Many ports got the aspect ratio wrong. \doom{} 95 which was to showcase Microsoft's Windows 95 graphics drivers low overhead was among the guilty. In its default setting, the 320x200 hosting window directly maps \doom{}'s core framebuffer. As a result, enemies look shorter, rocket explosions are oval, and everything else is distorted.}
\par
\cscaledimage{1}{doom_title_4_3}{\doom{} in 4:3 aspect ratio as presented on a 1993 PC monitor}
