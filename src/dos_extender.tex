\section{Dos Extender}
\tcode{dos4gw.txt}

the CPU is switched to Protected Mode\footnote{Some driver implementation relied on I/O ports}, a page is swapped between the EMS\footnote{Expanded Memory Specification} and the HMA\footnote{High memory area}. The CPU is then switched back to Real Mode where the 16KiB page is now accessible. Given this mechanism, no executable code can be stored there, it is only useful to store data.\\
\par
\textbf{\underline{Trivia :}} Switching from Real Mode to Protected Mode is easy, you only need to set the Control Register bit 0 to 1. It can be done with 6 instructions:
\par
\acode{switch_to_protected_mode.asm}
\par
However, switching back from Protected Mode to Real Mode is impossible on a 286 since Intel designers never envisioned this would be needed. As a result the only way is to issue a reset instruction \cw{HALT} to the CPU which can take milliseconds to execute. The 386 does not have this problem since it recognizes correctly when \cw{CR0} bit 1 is set back to 0.