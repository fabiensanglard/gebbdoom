\section{Graphic assets}

\subsection{Sprites}
For a team of only two artists, to produce that much artwork was an immense undertaking. Weapons were animated when firing, monsters had to have eight poses depending on the viewing angle. Add the textures for the walls, ceiling, floor and all the "utility" art for the menus and random health/barrels.\\
\par
%They solved the problem with an interesting combinations of skills, hard work, money and guile.\\
Drawing all assets for a monster took a lot of time. Drawing it once facing the player was relatively easy. Drawing it seven more times at increasing angle (45$^{\circ}$, 90$^{\circ}$, 135$^{\circ}$, 180$^{\circ}$, 225$^{\circ}$, 270$^{\circ}$, and 315$^{\circ}$) was hell. To solve this problem they created a process never used before, leveraging both their artistic talent and the technological power of the NeXTDimension. First they drew their vision. Then they applied clay on a Mold Wooden Manikin to bring it to life.\\
\par


\scaledimage{0.48}{sprites/doomguy_sketch.png}
\scaledimage{0.5}{sprites/doomguy_clay.png}\\
\par

\begin{wrapfigure}[9]{r}{0.4\textwidth}
\centering
\scaledimage{0.4}{SonyHi8.png}
\end{wrapfigure}
Once the character was carved they could change the pose at will. They only had to connect a Hi8 Handycam Sony videocamera to the NextDimension. Placed on a spinner, the clay model was lighted and digitized from eight different viewing angles.\\
\par
The result was 32-bit truecolor which had to be transformed to the doom palette of 256 colors via a tool called the "Fuzzy Pumper Palette Shop". To complete the set of sprites, the artist digitally paint it.\\
\par
\fullimage{sprites/doomguy_fpps.png}\\
\par
The process was not without it own flaws since the clay dried and had a tendency to break instead of fold. Nontheless, seven Doom characters were built as sculptures for Doom \& Doom II. The first models - the Doomguy, baron of hell and cyberdemon -- were sculpted by Adrian Carmack. \fixme{others are ???} \\
\par
\trivia{Most models survived. Some are still in John Romero possession while others are visible at XXX.}\\
\pagebreak

\cfullimage{sprites/adrian_hellknight.png}{Adrian Carmack sculpting the Hellknight, working from his preliminary drawing.}
\par
\fullimage{sprites/hellknight_fpps.png}\\
\par




\par
\trivia{Some complicated models such as the SpiderDemon where commissioned to Gregor Punchatz to build the rest of the models (the arch-vile, mancubus, revenant and spiderdemon). This was not Gregor's first gig since he had worked on major Hollywood movies such as Robocop before.}
\par
\fq{The spider creature was made out of parts I had literally just found at hardware and hobby stores, pieces of Tupperware and PVC pipes. The main body started out as a sculpture, then a plaster mold was pulled from that. Then we made the armature to fit that mold, and then foam latex was injected inside the mould and put into an oven.\\
\par
Mastermind's legs pretty much only just moved, and his arms moved, but his mouth didn't move. As we went along, the other maquettes become full ball and socket armatures, so they had a full range of motion. In some ways, these stop-motion maquettes are easier to get right than they would be in CG. You don’t have to worry about how your skin is weighted on stop-motion model because it just sticks to the metal armature.}{Greg Punchatz, Interview by develop-online.net Feb 16, 2016}\\
\par

\fq{At one stage id offered me points on the backend to take \$500 off the price of one of the characters and I turned that down. It’s a painful lesson. But to be part of something that has left a long-lasting impression on the world is kind of crazy – people find out that I worked on Doom and it’s like I played on the Beatles’ White Album.}{Greg Punchatz, Interview by develop-online.net Feb 16, 2016}

\cfullimage{sprites/spiderdemon_model.png}{Notice the spinner, camera, and a virgin Wooden Manikin on the table}
\par
\fullimage{sprites/spiderdemon_model_fuzzy.png}





\section{Props}

\fullimage{props/chainsaw.png}\\

\par
\trivia{To help convert code between NeXT and DOS, they wrote a tool named \cw{removecontrolm} and \cw{unfuck}}
