\section{Watcom Compiler}
In 1993 the best compiler available on MS-DOS was irrevocably Watcom C 9.0\footnote{Editor's choice -- PC Magazine, April 1995}. Commercialized by Watcom International Corporation, the compiler not only featured i486 optimizations, it also came with a DOS Extender, \cw{DOS/4G} thanks to a partnership with Rational Systems.\\
\par
Watcom International Corporation found its roots in 1986 when a team of undergraduate students at the University of Waterloo in Canada developed a FORTRAN compiler. The small team of three made of Fred Crigger, Ian McPhee, and Jack Schueler improved the compiler to support multiple language and platforms.\\
\par
In 1993 it was impossible to open a Byte or PC Magazine without stumbling upon an ad for the version 9 of their compiler. \footnote{Watcom C first released was Watcom C 6.0. There was never a 1.0 or 2.0. This was a marketing coup aiming at its two competitors, Microsoft and Borland, who at the time commercialized version 5 of their products (Borland C++ 4 and YYY). Using version 6.0 made it look like Watcom was one release ahead of its competitors.}
\\
At a cost of \$639\footnote{Byte Magazine March 1992} (a standard price for a compiler at the time), it was not cheap but opened a door to a new world of possibilities for id Software. The DOS Extender would be the corner stone in the bridge between PC and NeXT. Borland C++ for example had limited Wolfenstein 3D to Real Mode programming where C code (in a major departure from the intent of the language) would not be portable due to new keywords like \cw{near}, and \cw{far}.\\
\par



\subsection{Speed}
Third player besides Microsoft and Borland.

Introduced in 1988 by , Watcom C Compiler ads found in BYTE and PC Magazine:\\
\par
\fq{WATCOM C/C++ will produce code which is at *least* 
twice as fast as your current 16 bit compiler, and more typically around 
five times as fast.}{rec.games.programmer}
premiere C compiler for high performance and reliability\\
Version 6 direct.\\
Alliance with DOS4GW.\\
only 8 people.\\
A few years later
PC Magazine ran a comparative study of a collection of C compilers.  They
hailed WATCOM C/C++ v9.5 for producing the fastest code (by a *wide* margin)
but criticized it for its klunky command line driven interface.\\

\subsection{Popularity}
id Software were not the only team to value Watcom solution. Many other studios enthrusted it with their code and as a result many emblematic software of the 90s were built with Watcom.\\
\begin{enumerate}
\item id Softare 
       \begin{enumerate}
       \item Doom
       \item Doom II
       \end{enumerate} 
\item Blizzard Entertainment 
       \begin{enumerate}
       \item Warcraft
       \item Warcraft II
       \end{enumerate}
\item Ken Silverman's BUILD Engine based games
      \begin{enumerate}
       \item Duke Nukem 3D
       \item Shadow Warrior
       \item Blood
       \end{enumerate}
\item LucasArts Entertainment Company
      \begin{enumerate}
       \item Full Throttle
       \item  The Dig
       \item Dark Forces
       \item Rebel Assault II       
      \end{enumerate}
\end{enumerate}
\par
