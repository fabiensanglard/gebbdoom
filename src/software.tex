\section{Source Code}
The source code of \doom was released on December 23, 1997, roughly four years after the commercial release of the game. More than twenty years later, the zip archive is still where it was originally placed, on id Software ftp server.\\
\par
\tcode{doom_src_zip_url.c}
\par
 In the long series of id software source code release\footnote{From 1993 to 2012, id Software released the code for all nine games it produced.}, \doom stands apart since what was released was not what was used to ship the game. What was open sourced was neither the MS-DOS nor NeXTSTEP version. There is a little bit of a back story about it.\\
 \par
 In early 1997, Bernd Kreimeier approached id Software with a business offer. He wanted to write a book explaining the internals of the game engine, how to compile, and how to modify it. The idea was to release the book along with the source code.\\
 \par
  People at id, especially John Carmack and John Romero thought it was a great idea. They promptly sent him the source code. Upon reviewing it Bernd realized he had to make a few important decisions. Between the development requirements and Dave Taylor ports there was code specific to many platforms. The engine could be compiled on no less than five operating systems. Linux, NeXTSTEP, IRIX SGI, and of course MS-DOS were supported. To make the code easier to understand, Bernd decided to pick one platform and delete everything unrelated.\\
  \par
  The ideal choice would have been the MS-DOS version. It was the dominant operating system and it was the version players had experienced the game with. However, there was a copyright issues. id Software had licensed an audio library, DMX, which code was proprietary and could not be included with the source. MS-DOS was a no-go.\\
  \par
   An other option would have been to release the NeXSTEP version which had seen the most usage and therefore more stable. However since NeXT had stopped manufacturing workstation in 1994 and sold less than 50,000 units over its lifetime this was also a dead end. Few people would have had the software to enjoy it. There was a second problem with NeXTSTEP because the sound and music systems had never been implemented. NeXTSTEP was also a no-go. The third available solution was the Linux build. And that is what Bernd picked.\\
\par
 As he was cleaning up the code from everything not related to Linux and writing the book at the same time, hardware and software kept on evolving. Things evolved faster than he could write and before he could finish, interest in Doom was decreasing in favor of newer engines such as Quake and Duke Nukem 3D.\\
 \par
  With the profitability of the project compromised the book was abandoned\footnote{Too bad since what surfaced, "A Brief Summary of DOOM style Rendering by Robert Forsman July and Bernd Kreimeier" was of great quality.}. With the blessing of id Software, Kreimeier released the Linux code he had cleaned up. This port became the base of hundreds of fork since.\footnote{The original MS-DOS code can largely be reconstructed thanks to Raven which was much less conservative about DMX overlaps. Large portions of previously censured system of the game such as \cw{i\_sound.c} and \cw{i\_ibm.c} can be found in Heretic, and Hexen source code.}.\\
 \par  


