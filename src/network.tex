\section{Network}
The early 90s predated the democratization of the internet and wifi by a decade. Connecting computers together was difficult and expensive. Even if you had the means, bandwidth and latency were abysmal. Most of the time, playing with friends meant getting all computers into the same room. Playing from the comfort of your room was extremely uncommon.\\
\par To connect, players had were three technology available: NullModem cable, Modem and LAN via Network cards.\\
\par









\subsection{Null Modem Cable}
The cheapest way and what most people used was the \$20 cable known as "Null Modem" which could be plugged in each PCs COM port. The cable offered no modulation at all (hence the name). Only two players could participate but back then it was so new and cool that it felt like the most amazing thing in the world.\\
\par
Nowadays it is something natural and almost the bare minimum to expect connectivity. But back in the early 90s, to pack your 50 pounds machine (including the CRT) on your bike, making it to a friend's place alive, plug the cable, start \doom and finally see your character move on the other computer screen was an indescribable feeling. To realize the machines were actually communicating felt unreal and almost magical.\\
\par 
\cscaledimage{0.5}{nullmodemcable.png}{NullModem cable}








\subsection{BNC LAN}
To play with more than two players was substantially more difficult. Besides the relatively easy financial burden to pay for the equipment, you had to overcome the much more difficult task to convince a parent to let four teenagers come to their house where they would scream all night.\\
\par
The famous saying: "Shame on me if you fool me once, shame on you if you fool me twice" is often rumored to have originated from betrayed mothers and fathers whom had been \textit{dommed} all night.\\
\par


\begin{wrapfigure}[7]{r}{0.25\textwidth}
\centering
\scaledimage{.25}{BNC_T-piece.png}
\end{wrapfigure}

Leaving creative ways to ask for forgiveness aside, on the technical side a player had to plug a BNC network card via the ISA bus. The card had a BNC connector upon which was to be plugged a T-shaped connector. Each PC was connected to two other machines via 10-Base-5 coaxial cables. 
There was no central point in this type of networking, all machines involved in the networked formed a chain. At both ends the chain had to be plugged signal terminator in order to prevent an RF signal from being reflected back from each end, causing interference, or power loss.\\
\par
\begin{figure}
\centering
\scaledimage{0.7}{BNC_connector_50_ohm_male.png}
\caption{bla}
\end{figure}

\par
Once physically connected, there was no IP configuration required. The Network card MAC address were enough to run the Ethernet protocol which all games used.\\
\pngdrawing{10base5bncConnector}{10-Base-5 BNS Connector.}
\par

\trivia{Adding new machine on the network meant all other machines lost connectivity. Everybody remember this one friend who was always late to the LAn and forcing everybody to disconnect so he could join.}










\subsection{Modem}
The most fortunate player were able to afford the luxury of networking from home. That was very expensive since they not only had to pay for a modem and the service of an Internet monthly but they also had to pay for the time spent online.\\
\par 
A time before broadband, modem used phone landlines to connect to the Internet provider. These landlines were owned by telecom companies and you literally had to pay for each minutes connected, whether data was transfered or not.\\
\par
2h/day.\\
Price of model + internet + landline. \\
\par
The technology was still in its infancy. On top of costing you a fortune, modems were frankly ugly.\\
\par
\cscaledimage{0.9}{robotic28-8.png}{US Robotic 28.8 bauds modem. The top of the line in 1994.}

 Upon establishing the initial handshake during dial up, the modem speaker was kept turned on. An attuned ear could easily recognize the different phase of the connection startup. V.X bis transaction, speed negotiation, echo canceller disabling, modulation modes selection, all together formed an unforgeable melody which this book cannot do justice to but reader should definitely research it.\\
 \par 
\cfullimage{spectrogram2.png}{The spectrogram of V.34 modems handshake (Courtesy: Oona R\"{a}is\"{a}nen)}
\par
 \pagebreak
 \par


All along the 90s the bandwidth kept on steadily improving. Upon \doom release most modem were capable of kbit/s. Those who downloaded the shareware version in December 1993 had to wait 20 minutes to collect the (2,166,955 bytes) of the zip archive.\\
\par

 \begin{figure}[H]
\centering  
\begin{tabularx}{\textwidth}{ L{0.2} L{0.3} L{0.5}}
  \toprule
  \textbf{Year} & \textbf{Version} & \textbf{Bandwidth} \\
  \toprule 
   
    1990 & V.32 & 9.6 kbit/s \\
    1991 & V.32bis &  14.4 kbit/s \\
    1994 & V.34 & 28.8 kbit/s \\
    1995 & V.34 & 33.6 kbit/s \\
    1996 & V.90 & 56.0/33.6 kbit/s\\
    1999 & V.92 & 56.0/48.0 kbit/s\\
   
   \toprule
\end{tabularx}
\caption{Modem bandwidth evolution over the 90s.}
\end{figure}



\par
Modem issued commands using Hayes command set to control the line. \\

\par
 \begin{figure}[H]
\centering  
\begin{tabularx}{\textwidth}{ L{0.2} L{0.15} L{0.65}}
  \toprule
  \textbf{Modem A} & \textbf{Modem B} & \textbf{Comments} \\
  \toprule 
   
    ATDT15551234 &	&	Modem A issues a dial command: AT-Get the modem's ATtention; D-Dial; T-Touch-Tone; 15551234-Call this number\\
    \toprule 
      & RING	& Modem A begins dialing. Modem B's phone-line rings, and the modem reports the fact.\\
      \toprule 
    & ATA	& Modem B issues answer command.\\
    \toprule 
    CONNECT	& CONNECT	& The modems connect, and both modems report "connect"..\\
    abcdef	& abcdef	& When the modems are connected, any characters typed at either side will appear on the other side.\\
    \toprule 
    & +++	& Modem B issues the modem escape command.\\
    \toprule 
     OK &	& The modem acknowledges it.\\
    \toprule 
    & ATH	& Modem B issues a hang up command.\\
    \toprule 
    NO CARRIER &	OK	& Both modems report that the connection has ended. Modem B responds "OK" as the expected result of the command; modem A says NO CARRIER to report that the remote side interrupted the connection.\\
   \toprule
\end{tabularx}
\caption{Modem bandwidth evolution over the 90s.}
\end{figure}
\par
The fragility of early connection were subject of many jokes.\\

\par
\trivia{An humourous way to end a message was to type some garbage followed by NO CARRIER. ("Hey! Wait! Don't pick up the ph\{\#`\%\$\{\%\&`+'\$\{`\%\&NO CARRIER").}