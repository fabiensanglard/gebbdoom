\section{Sound System}
Thanks to its aggresive marketing, superior technology, and cheaper cards, Creative Labs dominated the market of sound cards. In order to hope surviving the market, any new cards had to label itself "Creative Labs compatible". Up to that point, sound cards had combined FM synthetized used for music and a DSP to playback digitized sounds in 44Khz, 8-bit per sample, stereo quality.\\
\par
 The early 90s would be the theatre of the last wave of innovation for PC audio gaming and saw the extinction of a previously key manufacturer named AdLib. For \doom, two cards constituted the top of the line. The Sound Blaster 16 by Creative Labs and the Gravis Ultrasound by Advanced Gravis Computer Technology Ltd\footnote{An other Canadian company!}.\\
\par
\subsection{Sound Blaster 16}
 In June 1992, with the release of the Sound Blaster 16, Creative Labs solved the problem of audio for games on PC forever with a card capable of CD quality playback, 44Khz, 16-bit sample on stereo.\\
\par
\fullimage{SoundBlaster_16_ASP_CT1740.png}
\par
Solving the audio problem turned out to be a problem itself for Creative Labs. Innovation for audio professional were not enough to drive high volumes. A few attempts to innovate for games with ASP and EAX technologies were made but there were of little appeal to consumers ears. As audio chips became cheaper to make and with the technical requirements stagnating, manufacturers started to provide audio capability built-in the motherboard.\\
\par
For a time the only devices able to provide an extra IDE connectors, audio cards survived by being bundled with CD-ROMs. That was not enough to save them. Within ten years the market of audio cards dissapeared.\\
\par
\drawing{sb16}{}
\par









\subsection{Gravis UltraSound}
Gravis Computer Technology originally built what was universally accepted as the best PC joypad, the Gravis PC GamePad. Strong with the cashflow they decided to enter the sound card market with an audatious and innovative card. The Gravis Ultrasound (nicknamed GUS) was released in 1992.\\
\par
The GUS offered Sound Blaster 16 compatible music and digitized sound playback. On top of that the card had a capability like no other cads one the market. On top of being able to synthetic music with a FM chip like the Sound Blaster did, it was able to used digitized instrument sounds. The technology named "Wavetable Synthesis" achieved an audio quality far superior to its competitors.\\
\par
\fullimage{Gravis_UltraSound_PnP_Pro_V1.png}
\par
Components list here: GF1 chip\\
Ram: 256 KiB (2x128 KiB which could be replaced with 2x512KiB). here\\
and here\\
\drawing{gravis}{}
\par
The cost of this technology was twofold. First the card needed RAM to store the sample. Since samples are much more voluminous than sin equations, it shipped with 256 KiB and could be boosted to 1 MiB. Second, the card needed sample which Gravis provided many of. The card installation came with 12 MiB of sound samples which was enormous at the time (The full version of \doom was 12 MiB as a matter of comparaison).\\
\par
The card was an instant success for the demomaker (who lover the quality of wavetables) and developed a cult following. Despite its superior quality, players were hard to convince to fork the \$200, more than \$129 of the Sound Blaster 16. The SB emulation had some issues and the card also was infortunate to be released shortly before the RAM shortage of 1993/1994. Its reliance on low RAM price was partly reponsible for its demise. By beting the company's future on its sound card, Gravis almost went bankrupt\\
\par
 \doom was one of the few title to support the Gravis Ultra Sound, this book cannot do it justice but listening to the guitar of  "At Doom's Gate" rendered on a GUS is quite an experience which make the FM version pale in comparaision.\\

\par
BALBLA\\
.PAT files are GUS instrument files, or PATCH files.