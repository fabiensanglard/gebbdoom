\begin{wrapfigure}[3]{r}{0.4\textwidth}{
\centering \scaledimage{0.4}{snes_logo.png}}
\end{wrapfigure}
The Super Nintendo Entertainment System was released in 1990 in Japan and the following years in USA and Europe. It was the 16-bit replacement to the 8-bit NES. In Japan, the Super Fami-Com ("FAMIly COMputer") was an instant success with the initial shipment of 300,000 units sold out within hours. The frenzy was such that the government requested Nintento to release its future systems on weekends to avoid further disturbances.\\
\par
Nintendo had established a merciless system to ensure quality of the games. Publisher were only allowed five games per year. To make sure this rule was enforced, only Nintendo was allowed to produce cartridges. Publisher had to buy them from Nintendo. To make sure everybody played by the rules (and also protect games from being copied) the SNES looked for a CIC lockout chip before a game was allowed to start. It was a powerful mechanism which was only cracked as the SNES had been replaced.\\
\par
During its nine years lifespan\footnote{The Super Nintendo was discontinued in 1999.} 721 games were published among them both technical and financial success such as Super Mario World, Zelda III, Mario Kart, Z-Zero, Super Metroid, or Donkey Kong Country. Having sold close to 50 million units it is arguably one of the best console of all times both in terms of sales and catalog.\\
\par
\cfullimage{consoles/SNES.png}{The Super Famicom by Nintendo.}
\par
From a technical standing point, the SNES exceled at 2D. Its 65C816 3.58 MHz CPU piloted a PPU Picture Processing Unit to generate video large sprites using up to 256 colors at a resolution of 256x240. On the audio side a powerful combo made of an 8-bit SPC700, and a 16-bit DSP with 64 KiB of dedicated SRAM.\\
\par

Needless to say that despite it impressive 2D sprite engine capabilities, \doom was never doing to fly on a SNES. As fate would have it, a small UK firm would make it happen.\\
\par
Argonaut Games, plc was a British video game developer founded by teenager Jez San in 1982. Although originally focused on the C64 market, they latter leaned toward the NES and then the SNES.\\
\par Around 1993, the gifted hackers managed to impresse Nintendo\footnote{Source: \cw{eurogamer.net} article "Born slippy: the making of Star Fox".} and landed a contract for a game nicknamed "NesGlider" which would later become the legendary "Starfox". The project was to become a features real-time 3D graphics. As gifted and creative programmers as they were, there was only that much they could do with the hardware. What they needed was more raw power.\\
\par
\fq{I told them that this is as good as it's going to get unless they let us design some hardware to make the SNES better at 3D. Amazingly, even though I had never done any hardware before, they said YES, and gave me a million bucks to make it happen.}{Jez San}.\\
\par
Jez's team made good usage of Nintendo's investment. They came up with something originally codenamed Mathematical Argonaut Rotation I/O, or "MARIO". The "Super FX" as it would be later marketed was so powerful, it was joked the Super NES was just a box to hold the chip. Since there was no way to modify the console, the chip was soldered on each new game cartridge which increased MSRP significantly.\\
\par
The GSU\footnote{Graphics Support Unit} as its technical name would be\footnote{I agree. These are a lot of names for a chip.} had a simple design based on a RISC processor running at 21.4 Mhz. Its sole purpose was to draw polygons to a frame buffer in the RAM sitting next to it where its data would be periodically transfered to the main system RAM via DMA.\\
\par




\fullimage{snes_cartridge.png}
\par
To open a \doom game cartridge reveals all comoponents previously discussed. \circled{1} The 32-bit GSU-2, \circled{2}the 512 KiB framebuffer where the GSU writes, \circled{3} ROM where the game code is stored, \circled{4} XXX, \circled{5} the copy protection CIC chip.\\
\par
The first generation of GSU-1 powered five games: Dirt Racer, Dirt Trax FX, Star Fox, Stunt Race FX and, Vortex. The GSU-2 was the same processor with extra pins soldered to the bus to increase the size of the addressable framebuffer. It was used in three games Doom, Super Mario World 2: Yoshi's Island, and Winter Gold\footnote{Star Fox 2 was finished but canceled at the last minute to not amped the imminent release of the Nintendo64.}
\rawdrawing{snes_board}



But as usual hackers found a way. Initial claim: 10x faster. Result. 40x times faster. Development was backward. They added features based on Nintendo needs. Not to make a generic processor.\\
\par

\trivia{Some passionate fans have managed to collected all 791 games of the SNES catalog. Seeeing them on a shelve is impressive. You can usually spot \doom cartridge from 20 feet away. Only three games were ever allowed to not be made of the standard gray. Two were red: "DooM" and "Maximum Carnage" while "Killer Instinct" was black.}
\pagebreak


\drawing{snes_cartridge}{SNES 721 games library. Zelda stands apart. Because Zelda stands apart.}
\par
\rawdrawing{snes_cartridge2}


\subsection{Doom On Super Nintendo}

\par
\begin{wrapfigure}[11]{r}{0.25\textwidth}
\centering
\scaledimage{0.25}{superfx.png}
\end{wrapfigure}
When Sculptured Software was tasked to port \doom to SNES, its head of development, Randy Linden, immediately decided to use Argonaut's Super DX 2 chip\footnote{The Super FX 2 was almost identical to the Super FX 1 except for a tighter packing allowing more bus address pin resulting in more ROM and more RAM.}. However even with the hundreds of polygons per seconds the chip could render, the original quality of PC \doom was still unattainable.\\
\par
 Randy Linden ended up writing a new renderer which he called "Reality engine". Many sacrifices had to be done to make the game run with the most notable being no textured flat, duplicated horizontal lines, no sound propagation and always front facing enemies.\\
\par Even with all these restriction, the Reality Engine still had to reduce the rendering area. Out of the native 256 x 224 resolution, only 216x176 were actually drawn and only 216x144 for the 3D canvas. With horizontal lines doubles, it means Reality Engine was actually rendering at 108x144 to reach an average of 10 fps which was still a remarkable achievement.\\





\fq{DOOM was a truly ground-breaking title and I wanted to make it possible for gamers without a PC to play the game, too. DOOM on the Super Nintendo was another one of those programming challenges that I knew could be accomplished.\\
\par

I started the project independently and demo'd it to Sculptured Software when I had a fully operational prototype running. A bunch of people at Sculptured helped complete the game so it could be released in time for the holidays.\\
\par
The development was challenging for a few reasons, notably there were no development systems for the SuperFX chip at the time. I wrote a complete set of tools — assembler, linker and debugger — before I could even start on the game itself.\\
\par
The development hardware was a hacked-up StarFox cartridge (because it included the SuperFX chip) and a modified pair of game controllers that were plugged into both SNES ports and connected to the Amiga's parallel port. A serial protocol was used to communicate between the two for downloading code, setting breakpoints, inspecting memory, etc.\\
\par
I wish there could have been more levels but unfortunately the game used the largest capacity ROM available and filled it almost completely. I vaguely recall there were roughly 16 bytes free, so there wasn't any more space available anyway! However, I did manage to include support for the SuperScope, Mouse and XBand modem! … Yes, you could actually play against someone online!}
{Randy Linden (Interview with gamingreinvented.com)}
\par
Almost no censorship from Nintendo except for removing red blood.
\par

\scaledimage{0.3}{doom_red_cartridge.png}

\cfullimage{consoles/snes/wide.png}{}
\par
\cfullimage{consoles/snes/enemy.png}{}

\pagebreak


   % Super FX2:
   % Released 1993\\
   % GSU: 23 MHz\\
   % VRAM: 32 KiB\\
   
   % SNES:\\
   % Released 1990\\
   % CPU: 5A22 3.58 MHz\\
   % Audio: SPC700\\