The success of the PC version and its mind numbing sales figure made it an extremely desirable title for any console publisher of the early 90s. From 1994 to 1997, \doom was ported to all six major systems of the era.\\
\par
% \begin{itemize}
%    \item 1994: Atari Jaguar and the Sega 32X.
%    \item 1995: Super Nintendo and Sony Playstation.
%    \item 1996: 3DO.
%    \item 1997: Sega Saturn.
% \end{itemize}
% INSER ARRAY HERE: Console, Year, Authors, Avg framerate, price, CPUS, RAM.
% \bigskip

% \begin{figure}[H]
% \centering  
% \begin{tabularx}{\textwidth}{ L{1}  R{1} R{0.5} R{0.5} R{0.5} R{0.5} }
%   \toprule
%    \textbf{Year} & \textbf{System} & \textbf{\#CPUs} & \textbf{RAM} & \textbf{Resolution} & \textbf{Price} \\
%   \toprule 
%             1994 & Atari Jaguar.   &  5              & 2 MiB.       &  320x240            & BLA\\
%             1994 & Sega 32X        &  3              & ISA-8        &  8                  & BLA\\
%             1995 & Super Nintendo  &                 & ISA-16       & 13                  & BLA\\
%             1995 & Sony Playstation&                 & VLB          & 24                  & BLA\\
%             1996 & 3DO             &                 & VLB          & 24                  & BLA\\
%             1997 & Sega Saturn     &                 & VLB          & 24                  & BLA\\
%    \toprule
%  \end{tabularx}
% \caption{Ports of \doom to video game console from 1995 to 1997}
% \end{figure}

This time period is known as the "console war" during which the generation of a system was associated to its "bitness". Third generation, 8-bit based, NES and Sega Master System had dissapeared. The fourth, 16-bit generation, made of Nintendo's SNES, TurboGrafx-16 and Sega's Genesis was reaching end of life. The fifth, 32-bit, generation 
was starting to appear with Sony's Playstation1 and Sega Saturn with marketers trying to play on words with systems branded 64-bit like the Nintendo64 or the Jaguar\footnote{After that, consumers either got tired or the "bits" or realized the silliness of the whole nomenclature. Bitness was forgotten.}.. In retrospect, it was a rich period, propice to hardware innovation which contrast vividly when compared to 2018 uniform world of Sony vs Microsoft where system only differ by their name.\\
\par
The architecture of the engine based on a core with "only" sub-system to implement may lead to conclude it was an easy task to get \doom to run on a console. This feeling could not be further from the truth. From design trade-off due to restricted resources, crazy schedules, to anecdote of a lone hero programmer trying to make the everything fit in, all versions have an unique and rich story.\\
\par
From a technical standing point, the common problem to solve was to deal with less memory than originally intended. PCs minimum requirement was to have 4 MiB installed on the machine. It was of course not possible to ask customers to add more RAM to their console. Developers sometimes had to deal little as little as 512 KiB of RAM.\\
\par
The second difficulty was to deal with exotic hardware. The PC was designed around one "big iron" CPU while consoles were a made of a constellation of processors.












\section{Jaguar (1994)}
\subfile{port_jaguar} 




\section{Sega 32X (1994)}
\subfile{port_x32} 






\section{Super Nintendo (1995)}
\subfile{port_snes}








\section{Playstation 1 (1995)}
\subfile{port_psx}










\section{3DO (1996)}
\subfile{port_3do}




\section{Saturn (1997)}
\begin{wrapfigure}[8]{r}{0.25\textwidth}{
\centering \scaledimage{0.25}{saturn_logo.png}}
\end{wrapfigure}
Replacement for the insanely popular yet aging Nemesis.
The Sega Saturn was rapidely outgunned by its competitors. It brought back the dominance on the japanase market which Sega wanted so much but also precipited the collapse of its international empire.\\
\par
 Saturn got a good start with Daytona USA and Virtua Fighter being impressive. However Tekken and Ridge Racer on PlayStation two weeks later put the Saturb to shame. Putting Ridge Racher and Daytona next to each others revealed the low framrate, polygon popup issues and letter-boxed presentation\\
\par
\fullimage{consoles/Saturn.png}\\
\par
Rushed release to undercut PlayStation. Developers not given enough time (only six games at launch). Even suppliers were taken by surprise. 1995 saw improvement with Virtua Cop and Virtua Fighte 2 but mostly in Japan where FV was more popular than Sonic, Mario or Tetris\footnote{Source: 2006 poll for Top 100 games.} But PlayStation sold 3x more. In June 1996 an other competitor Nintendo64 achieved the Saturn.
\par
The fatal mistake of the Sega Saturn is that it was not well equipped for 3D gamnes and 3D games were all the rage.
\par
Biggest mistake: "designing the Saturn as a modified last-generation 2D system when clearly 3D was going to be the next big thing".\\
https://www.doomworld.com/forum/topic/86671-dissecting-sega-saturn-doom/\\
\par
JIMSDOOM.WAD, obviously named after Jim Bagley.\\
The wad is a 1:1 copy of PSXDOOM.WAD\\
John Carmack shut down their original plan to use a hardware-accelerated renderer\\
\par
See author comments: https://www.doomworld.com/forum/topic/86671-dissecting-sega-saturn-doom/?page=2\#comment-1580057\\


\fullimage{affine_texture_mapping/post_far.png}\\
\fullimage{affine_texture_mapping/post_near.png}\\
% \par
% \fullimage{affine_texture_mapping/door.png}\\
Two SH-2 and two VDP chips plus Motorola 68EC000. 

\fq{Difficult to program, it used quads and messy 3D maths}{David Shea, Alien Trilogy developer}
Had to be programmed straight to the metal.\\
Retro-Gamer-134 is a gold mine of information.\\

\fq{When I started the project, I had to do a demo for id Software to approve. I started by extrating all the levels and audio and textures from the WAD files and made my own Saturn version of this, then got an early version of the renderer working using the 3D hardware. The got sent off and a couple days later I for a call from John Carmack, who stipulated that under no circunstances could I use the 3D harware to draw the screen. I had to use the processor like the PC. Thankfully I enjoy challenges, so it turned out to be a really enjoyable project, using both SH2s to render the display like the PC did it, using the 68000 to orchestrate them both.
\par
However, it kneecapped the game and the speed-framerate suffered greatly.}{Jim Bagley for RetroGamer \#134}

\par
Years later, in 2014, Carmack had reconsidered.\\
\par
\fq{I hated affine texture swim and integral quad verts, but in hindsight, I probably should have let experiment.}{John Carmack}


\par 
\fixme{Watch https://www.youtube.com/watch?v=784MUbDoLjQ before closing this chapter.}
\par
See: http://itrunsdoom.tumblr.com/archive\\
\par



